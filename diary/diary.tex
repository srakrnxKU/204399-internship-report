\section*{4/6/2562}

เนื่องจากเข้าทำงานเป็นวันแรก จึงต้องจัดสถานที่ทำงาน และทำงานต่อจากที่ได้รับมอบหมายก่อนการฝึกงาน

งานที่ได้รับมอบหมายโดยคร่าวคือการวิเคราะห์สภาวะความง่วงในคน โดยศึกษาจากกลุ่มเป้าหมายของพนักงานบริษัท
ซึ่งอาจารย์ที่ปรึกษาให้สิทธิ์ในการกำหนดแนวทางการวิเคราะห์ได้โดยอิสระ อย่างไรก็ตามงานของการวิเคราะห์ความง่วง
โดยตั้งต้นนั้นมักใช้การวิเคราะห์ภาพจากดวงตา (gaze monitoring) ซึ่งใช้วันนี้ในการหางานวิจัยตั้งต้น

นอกจากนี้ยังศึกษาแนวทาง ข้อกำหนด และมาตรฐานจริยธรรมในการทดลองภายในมนุษย์ (human subject research)

\section*{5/6/2562}

ศึกษาแนวทางในการทำ eye gazing ตามหนังสือที่ได้รับมอบหมาย และนำเสนองานวิจัยต่อจากที่เลือกจากเมื่อวาน

ปรับแก้แนวทางในการวิจัย และได้รับมอบหมายให้ออกแบบวิธีการทดลองโดยคร่าว
หารือกับทีมโปรแกรมเมอร์ว่าด้วยซอฟต์แวร์สำหรับการทดลอง

\section*{6/6/2562}

ศึกษาอุปกรณ์สำหรับติดตามดวงตา (Gazepoint) ก่อนจะพบว่าอุปกรณ์มีข้อจำกัดในการทำงานบางส่วน ทำให้ไม่สามารถดึง
ภาพดวงตาออกมาใช้ในโปรแกรมภายนอกได้ และติดต่อกับผู้ผลิตอุปกรณ์เพื่อหารือความเป็นไปได้ในการดึงภาพดวงตา

ศึกษาการใช้ Pytorch ในการทำการเรียนรู้เชิงลึก (deep learning) แทนที่ Keras

\section*{7/6/2562}

เปลี่ยนแนวทางการทำวิจัยด้วยข้อจำกัดของอุปกรณ์ มาเป็นการทำวิจัยบนกล้ามเนื้อตา (EOG) ค้นคว้าและทบทวนวรรรณกรรม
ที่เกี่ยวข้องกับงาน

ทำแบบทดสอบสำหรับบทเรียนจริยธรรมการวิจัยในมนุษย์จนอยู่ในเกณฑ์ได้รับประกาศนียบัตรผ่านการอบรม