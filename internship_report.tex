\documentclass[16pt,a4]{internshipreport}

\usepackage{lipsum}  
\usepackage[nottoc]{tocbibind}
\usepackage{appendix}
\usepackage{float}
\usepackage{wrapfig}
\usepackage{amsmath}
\usepackage{hyperref}
\usepackage{tabularx}
\usepackage{hhline}
\usepackage{tikz}
\usepackage{framed}
\usepackage{comment}
\usepackage{soul}
\usepackage{environ}
\usepackage{cite}
\usepackage{tabularx}
\usetikzlibrary{backgrounds}

\NewEnviron{redacted}{
    \begin{framed}
        {
            \begin{center}
            \textit{(ข้อมูลปกปิด)}
            \end{center}
        }
        \phantom{
            \begin{minipage}{\textwidth}
            \BODY
            \end{minipage}
        }
    \end{framed}
}


\newcommand{\inlineredacted}[1]{\fbox{\textit{(ข้อมูลปกปิด)}}}

\title{การวิจัยในโครงการตรวจสอบความง่วงด้วยตัววัดชีวภาพ}
\entitle{Research Assistant in Biomarkers-based Fatigue Detection}
\location{สถาบันวิทยสิริเมธี (VISTEC)}
\date{2562}
\author{ศิระกร ลำใย}
\studentid{5910500023}

\begin{document}

\maketitle

\chapter*{บทคัดย่อ}
\addcontentsline{toc}{chapter}{บทคัดย่อ}

การฝึกงาน ณ สำนักวิชาวิทยาศาสตร์และเทคโนโลยีสารสนเทศ สถาบันวิทยสิริเมธี ในระหว่างวันที่ 4 มิถุนายน 2562 ถึงวันที่ 31 กรกฎาคม 2562 นั้นเป็นไปตามข้อบังคับของหลักสูตรวิศวกรรมคอมพิวเตอร์ คณะวิศวกรรมศาสตร์ มหาวิทยาลัยเกษตรศาสตร์

หัวข้อในการฝึกงานครั้งนี้ คือการวิจัยความง่วงและความเมื่อยล้าจากการทำงานออฟฟิซ โดยมุ่งหวังการออกแบบการทดลองเพื่อสร้างความง่วง และหาวิธีการตรวจวัดระดับความง่วงนั้นด้วยข้อมูลชีวภาพ (biomarkers) นอกจากนี้ยังได้รับมอบหมายให้ช่วยทีมวิจัยการสั่งงานคลื่นสมองต่อเนื่องในการเขียนวารสารวิชาการ

ผลลัพธ์ในช่วงฝึกงาน คือทำการทดลองเก็บข้อมูลรอบแรก เพื่อหาความสัมพันธ์ระหว่างข้อมูลชีวภาพและระดับความง่วงในขั้นเบื้องต้น และเพื่อนำข้อมูลมาปรับปรุงการทดลองในอนาคต และบทความวิชาการว่าด้วยการสั่งงานคลื่นสมองต่อเนื่อง อยู่ระหว่างการตอบรับตีพิมพ์ในวารสาร IEEE Access

\chapter*{กิตติกรรมประกาศ}
\addcontentsline{toc}{chapter}{กิตติกรรมประกาศ}

ถ้าจะมีอะไรที่การฝึกงานครั้งนี้สอนให้ข้าพเจ้ารู้และมั่นใจ คงหนีไม่พ้นความจริงที่ว่าเราอยู่ในโลกที่ขับเคลื่อนด้วยความเชื่อ--และความเชื่อเหล่านั้นสร้างความเปลี่ยนแปลงทั้งในระดับที่เล็กจนเราลืมใส่ใจ และใหญ่จนเราไม่คิดฝันว่าจะเกิดขึ้นได้

ประการแรก ข้าพเจ้าขอขอบพระคุณครอบครัวที่เชื่อในตัวของข้าพเจ้า และเชื่อในทางที่ข้าพเจ้าเลือกเดิน ด้วยการให้อิสระทางความคิด แน่นอนว่าความเชื่อเหล่านั้นมีค่าอย่างยิ่ง และทำให้ข้าพเจ้าเติบโตมาในสภาวะที่กล้าตัดสินใจโดยไม่ลังเลหากไม่จำเป็น

ขอขอบพระคุณอาจารย์ธนาวินท์ รักธรรมานนท์ สำหรับความเชื่อในตัวนิสิตคนหนึ่งมากพอที่จะแนะนำให้เข้าฝึกงานที่นี่เมื่อปีที่แล้ว และขอขอบพระคุณอาจารย์ธีรวิทย์ วิไลประสิทธิ์พร สำหรับความเชื่อในตัวข้าพเจ้าที่มากพอ จนได้มาเป็นส่วนหนึ่งของสถาบันที่มุ่งเน้นความเป็นเลิศทางวิชาการ แม้จะเป็นเวลาไม่นานก็ตาม

ขอขอบพระคุณอาจารย์จากคณะวิศวกรรมศาสตร์ มหาวิทยาลัยเกษตรศาสตร์ หลายคนต่างมีความเชื่อในรูปแบบที่ต่างกันไป, ความเชื่อเหล่านั้นถ่ายทอดออกมาเป็นความตั้งใจ ความมุ่งมั่น และทัศนะอันหลากหลายยิ่ง: อาจารย์จิตร์ทัศน์ ฝักเจริญผล, อาจารย์ชัยพร ใจแก้ว, อาจารย์ธนาวินท์ รักธรรมานนท์, อาจารย์ภารุจ รัตนวนพันธุ์, อาจารย์จเร เลิศสุดวิชัย, อาจารย์กาญจนพันธุ์ สุขวิชชัย และท่านอื่นๆ ที่มิอาจเอ่ยนามได้หมด

ขอบคุณในความเชื่อที่จะเปลี่ยนโลกด้วยงานวิจัย ของอาจารย์จากสำนักวิชาวิทยาศาสตร์และเทคโนโลยีสารสนเทศ: อาจารย์สรณะ นุชอนงค์, อาจารย์ธีรวิทย์ วิไลประสิทธิ์พร, อาจารย์ปรเมษฐ์ มนูญพงศ์, อาจารย์ศุภศรณ์ สุวจนกรณ์ รวมถึงอาจารย์โชคชัย เลี้ยงสุขสันต์ คณบดีคนแรกของสำนักวิชา และขอขอบคุณบุคลากร นิสิต และนิสิต-นักศึกษาฝึกงานของสำนักวิชาฯ ที่พร้อมผลักดันความเชื่อนี้ในสภาพแวดล้อมการทำงานที่วิเศษ เปี่ยมไปด้วยพลังขับเคลื่อนและกำลังใจ: พี่เอ็ม, พี่น้ำเงิน, พี่เติร์ด, พี่ไอซ์, พี่แก๊ป, พี่ก้อง, พี่เจปัง, พี่ธนัท, พี่โอ, พี่อ้น, พี่หมีพูห์, บิล, พี่ลิลลี่, พี่บาล์ม, พี่แทน, พี่ฉัตร, พี่มินท์, พี่ส้ม, พี่พลอย, ไททัน, พี่แบงค์, ออฟ, มานีชา, เบสท์, ปาล์ม, เจ, พี่นานาไอซ์ และที่อาจไม่ได้เอ่ยนาม ทั้งนี้บุคลากรจำนวนไม่น้อยในสถาบันแห่งนี้เป็นผลผลิตแห่งความภาคภูมิใจของภาควิชาวิศวกรรมคอมพิวเตอร์ มหาวิทยาลัยเกษตรศาสตร์: พี่ต้า, พี่วิท, พี่เติ้ล, พี่เบญ, พี่ทิน, พี่บอส, พี่จุ๊บ, พี่นัท และต้อง

ขอขอบคุณเพื่อนนิสิตมหาวิทยาลัยเกษตรศาสตร์สำหรับความเชื่อที่มีให้ และขอบคุณยิ่งกว่าสำหรับทุกคำพูดที่บอกให้เชื่อในตัวเอง: รวิส, อ้น, กิต, รล, เบนซ์, มอร์แกน, นัท, นิว, เปรม, จุ้ย, ป่าน, แผว, บัว, เติ้ง, พร้อมกับนิสิตในรุ่น และในชั้นปีอื่นทุกคน

ขอขอบคุณสมาชิกรวมถึง ``แขกรับเชิญ" แห่งกลุ่มวิจัยเชิงทฤษฎี สำหรับพื้นที่สำหรับการเชื่อในสิ่งที่ตัวเองเป็นอย่างไม่หวาดหวั่น: พี่เนยสด, พี่บาส, พี่จูน, พี่เช, พี่แปลน, พี่มะเหมี่ยว, พี่ชวิน, พี่หมูแดง, พี่บลู และขอขอบคุณมิตรสหายบนสื่อออนไลน์ทุกท่านที่ครั้งหนึ่งเคยได้แลกเปลี่ยนความเชื่อและความเห็นซึ่งกันและกัน

ในขณะที่ข้าพเจ้าขับเคลื่อนตัวเอง วงดุริยางค์ฟิลฮาร์โมนิกแห่งประเทศไทย, วงดุริยางค์ซิมโฟนีแห่งลอนดอน, เบอร์ลินฟิลฮาร์โมนิก, เดอะ คาร์เพนเทอรส์, เดอะ บีเทิลส์, แฟรงค์ ซินาทรา, บีเอ็นเคโฟร์ตี้เอท, ฟีเวอร์, ไปส่งกูบขส.ดู๊, ชาบลูส์ และหอศิลปวัฒนธรรมแห่งกรุงเทพมหานคร ก็คงขับเคลื่อนหลายอย่างด้วยความเชื่อคล้ายๆ กัน น่าดีใจที่ความเชื่อของบุคคลหรือหน่วยงานเหล่านี้ เป็นแรงสำคัญให้ข้าพเจ้าได้มีกำลังใจเชื่อหลายๆ อย่างต่อไป

ขอบคุณไททัน--สาณุรักษ์ ณัฐนิธิการัชต์ ที่ทำให้ข้าพเจ้ามั่นใจว่าความเชื่อใดๆ ก็ตามล้วนคุ้มค่ากับทุกความรู้สึก ทุกความมุ่งมั่น ทุกความตั้งใจที่ใส่ลงไป และขอบคุณแพค--ภัทรนันท์ ลิ้มอุดมพร ที่ทำให้ข้าพเจ้าได้รู้จักความเชื่อในฐานะสิ่งที่ทรงพลังมากกว่าที่ข้าพเจ้าจะจินตนาการได้ ขอขอบคุณจากใจจริง

การเกิดขึ้นของสำนักวิชาวิทยาศาสตร์และเทคโนโลยีสารสนเทศ สถาบันวิทยสิริเมธี นั้นเป็นไปได้ด้วยความเชื่อของบริษัทปตท. จำกัด (มหาชน) พร้อมบริษัทในเครือ และธนาคารไทยพาณิชย์ จำกัด (มหาชน)  ข้าพเจ้าขอขอบพระคุณในความเชื่อว่าการเกิดขี้นของสถาบันฯ จะเป็นแรงสำคัญในการผลักดันประเทศ และหวังว่าครั้งหนึ่ง ในฐานะผู้ที่มีความเชื่อนี้ร่วมกัน สิ่งที่ข้าพเจ้าได้สร้างไว้จะร่วมส่งผลให้ความเชื่อนั้นเป็นจริงในสักวัน

\vskip 20pt

\hfill\begin{minipage}
    {\dimexpr 5cm}
    \begin{center}
        นาย ศิระกร ลำใย\\
        ผู้จัดทำรายงาน\\~\\

        วันสุดท้ายของการปฏิบัติงาน\\
        31 กรกฎาคม 2562
    \end{center}
    \xdef\tpd{\the\prevdepth}
\end{minipage}


\tableofcontents

\listoffigures

\listoftables

\chapter{บทนำ}

\section{ความสำคัญและที่มา}
หลักสูตรวิศวกรรมศาสตรบัณฑิต สาขาวิชาวิศวกรรมคอมพิวเตอร์ หลักสูตรปรับปรุง พ.ศ 2556 ระบุให้ผู้เรียนทุกคนต้องเข้ารับการฝึกงาน เพื่อเพิ่มพูนประสบการณ์ในการเรียนรู้ที่ไม่อาจหาได้ในห้องเรียน และเป็นการฝึกทักษะของวิศวกรในการทำงานจริง

คณะวิศวกรรมคอมพิวเตอร์ และภาควิชาวิศวกรรมคอมพิวเตอร์ จึงกำหนดให้มีการเรียนการสอนในรายวิชา 01204399 หรือการฝึกงาน แบ่งเป็นการฝึกงานภาคฤดูร้อนสำหรับนิสิตที่ไม่ได้สหกิจ และฝึกงานต่อเนื่องในช่วงเวลาของภาคฤดูร้อนและภาคการศึกษาต้นของมหาวิทยาลัยสำหรับนิสิตที่สหกิจ จึงเป็นที่มาของรายงานเล่มนี้ซึ่งเป็นหนึ่งในข้อกำหนด/ข้อบังคับของการฝึกงาน

\section{วัตถุประสงค์การปฏิบัติงาน}
\begin{itemize}
    \item เพื่อเพิ่มพูนประสบการณ์ในการเรียนรู้ที่ไม่อาจหาได้ในห้องเรียน
    \item เพื่อพัฒนาทักษะการทำงาน การสื่อสาร และทักษะ soft skills อื่นๆ
    \item เพื่อเป็นการเตรียมตัวในการทำโครงงานวิศวกรรมคอมพิวเตอร์ และเป็นการเตรียมตัวเขียนวารสารทางวิชาการ
\end{itemize}

\section{ขอบเขต}
\begin{itemize}
    \item นำทีมวิจัยในส่วนการวิจัยความง่วงและความเมื่อยล้าจากการทำงาน
    \item ช่วยทีมสั่งงานคลื่นสมองแบบต่อเนื่องในการเขียนงานวิจัย
\end{itemize}

\section{ประวัติและรายละเอียดสถานประกอบการ}

\begin{figure}[H]
    \centering
    \includegraphics[width=0.8\textwidth]{images/vistec_v.jpg}
    \caption{อาคารหอสมุด สถาบันวิทยสิริเมธี}
\end{figure}

\textbf{สถาบันวิทยสิริเมธี (VISTEC)} เป็นบัณฑิตวิทยาลัย (graduate school) ซึ่งมุ่งเน้นความเป็นเลิศในการทำวิจัย ตั้งอยู่ในพื้นที่วังจันทร์วัลเลย์ (Wangchan Valley) และเขตนวัตกรรมระเบียงเศรษฐกิจพิเศษภาคตะวันออก (Eastern Economic Corridor of Innovation: EECi) เลขที่ 555 หมู่ 1 ตำบลป่ายุบใน อำเภอวังจันทร์ จังหวัดระยอง ก่อตั้งขึ้นเมื่อปี พ.ศ. 2558 โดยมูลนิธิพลังสร้างสรรค์นวัตกรรม ภายใต้การสนับสนุนเงินทุนจากบริษัทในกลุ่มของการปิโตรเลียมแห่งประเทศไทย (ปตท.)

VISTEC มุ่งเน้นการจัดการศึกษาด้านวิทยาศาสตร์ วิศวกรรม และเทคโนโลยี โดยมีศูนย์วิจัยวิทยาศาสตร์และเทคโนโลยีชั้นแนวหน้า (Frontier Research Center) ซึ่งเป็นศูนย์กลาง ในการเสริมสร้างความเข้มแข็งทางการวิจัย และให้การสนับสนุนด้านทุนการวิจัยแก่สถาบันฯ เป็นศูนย์รวมนักวิจัยที่มีความเชี่ยวชาญสูง ช่วยขับเคลื่อนการดำเนินงานด้านการศึกษา วิจัย การสร้างนวัตกรรม สร้างความร่วมมือทางด้านวิจัยกับสถาบันการศึกษา ภาคธุรกิจ ภาคอุตสาหกรรม
และหน่วยงานด้านการวิจัยวิทยาศาสตร์และเทคโนโลยี \cite{wiki-vistec}

\begin{figure}[H]
    \centering
    \includegraphics[width=0.8\textwidth]{images/brain_team_2019.jpg}
    \caption{ทีมวิจัย Interfaces, ห้องปฏิบัติการ BRAIN}
\end{figure}

\textbf{ห้องปฏิบัติการเบรน (Bio-inspired Robotics and Neural Engineering: BRAIN)}
ณ สำนักวิชาวิทยาศาสตร์และเทคโนโลยีสารสนเทศ สถาบันวิทยสิริเมธี มุ่งเน้นศึกษาการสร้างหุ่นยนต์ที่มีลักษณะร่วมกับกายวิภาค (anatomy) ของสิ่งมีชีวิต และใช้เทคโนโลยีจำพวก Machine Learning หรือ Deep Learning ในการจำแนก วิเคราะห์ และประมวลผลคลื่นสมองของมนุษย์ เพื่อสร้างส่วนติดต่อผู้ใช้ผ่านสมอง (Brain Controlled Interfaces: BCIs)

ลักษณะงานที่ได้รับผิดชอบจากห้องปฏิบัติการฯ เป็นงานของผู้ช่วยนักวิจัย (Research Assistant: RA) ซึ่งช่วยนิสิตระดับบัณฑิตศึกษาในการเตรียมการทดลอง ออกแบบ และพัฒนาเครื่องมือวัดผล ควบคุมการทดลอง และทดสอบสมมติฐานเพื่อตีพิมพ์องค์ความรู้ในวารสารวิชาการต่อไป

ที่ปรึกษาและผู้ควบคุมการฝึกงานในครั้งนี้ คืออ. ดร. ธีรวิทย์ วิไลประสิทธิ์พร หัวหน้าหน่วยวิจัย (Principal Investigator: PI) และมีระยะเวลาปฏิบัติงานประมาณ 2 เดือน กล่าวคือตั้งแต่วันที่ 4 มิถุนายน ถึง 31 กรกฎาคม 2562

\section{ประโยชน์ที่คาดว่าจะได้รับ}
เป็นการสร้างพื้นฐานในด้านงานวิจัย รวมถึงเตรียมพื้นฐานในการทำโครงงานวิศวกรรมคอมพิวเตอร์ โดยมีความมุ่งหวังจะต่อยอดงานดังกล่าวเป็นงานวิจัยตีพิมพ์ต่อไป

\chapter{ความรู้พื้นฐานและการทบทวนวรรณกรรม}
\section{ตัวชี้วัดทางชีวภาพ (Biomarkers)}
ตัวชี้วัดทางชีวภาพ (biomarkers) เป็นตัวบ่งชี้ต่อสภาวะต่างๆ ที่เกิดกับร่างกาย ซึ่งรวมถึงแต่ไม่จำกัดเพียงแต่สถานะการตื่น สภาพอารมณ์
หรือสัญญาณบ่งชี้ของโรค

\section{คลื่นสัญญาณชีวภาพ (Biosignals)}
คลื่นสัญญาณชีวภาพ (biosignals) เป็นคลื่นสัญญาณจากกระแสไฟฟ้าในร่างกาย ซึ่งสามารถตรวจวัดได้ด้วยวิธีการที่ต่างกันไป
และผลจากการตรวจวัดคลื่นแต่ละส่วนจะบ่งบอกซึ่งข้อมูลที่แตกต่างกันออกไปเช่นกัน

งานวิจัยของห้องปฏิบัติการเบรน มุ่งศึกษาคลื่นสัญญาณชีวภาพ Electroencephalography และ

\section{คลื่นไฟฟ้าสมอง (Electroencephalography)}
Electroencephalography หรือ EEG เป็นคลื่นที่เกิดจากการตรวจวัดกระแสไฟฟ้าของสมอง การตรวจวัดโดยมากไม่จำเป็นต้องทำการเจาะผิวหนัง
(noninvasive) โดยใช้อิเล็กโทรดนำไฟฟ้าอ่านคลื่นสมองจากกระโหลก

การตรวจวัดและใช้ข้อมูลจากคลื่น EEG ส่วนมากมุ่งเน้นการใช้ศักย์ไฟฟ้าที่ขึ้นกับเหตุการณ์กระตุ้นของผู้ถูกวัด
(Event Related Potential) กล่าวคือมุ่งสังเกตจุดสูงสุดและต่ำสุดของศักย์ไฟฟ้าของคลื่นสมอง
และหาความสัมพันธ์ระหว่างเหตุการณ์กระตุ้นและการเพิ่มขึ้นหรือลดลงของศักย์ไฟฟ้า

\begin{figure}[h]
	\centering
	\includegraphics[width=0.5\textwidth]{images/1020.png}
	\caption{ภาพตำแหน่งของการติดขั้วนำไฟฟ้า (electrode) ตามระบบ International 10/20}
	\hspace{\linewidth}
	\textit{รูปภาพประกอบโดยผู้ใช้ Tomaton124 \href{https://commons.wikimedia.org/wiki/File:21_electrodes_of_International_10-20_system_for_EEG.svg}{บนโครงการวิกิมีเดีย คอมมอนส์}
		ชิ้นงานเป็นสาธารณะสมบัติ}
\end{figure}

อย่างไรก็ตาม การใช้คลื่นไฟฟ้านั้นยังสามารถใช้ประโยชน์จากคลื่นส่วนอื่น อันได้แก่คลื่นส่วน Motor Cortex ซึ่งถูกกระตุ้นด้วยการ "จินตนาการ"
การขยับร่างกาย และการใช้คลื่นส่วน Vision Cortex เพื่อกระตุ้นการมองเห็น เช่นการใช้สัญญาณ SSVEP จากสมองส่วนท้ายทอยซึ่งจะสั่นพ้อง
กับการกระพริบของแสงในความถี่ที่ตามองเห็น

\section{คลื่นไฟฟ้าจากการขยับลูกตา (Electrooculography)}
Electrooculography หรือ EOG เป็นคลื่นไฟฟ้าที่เกิดจากการขยับลูกตา โดยหากมองลูกตาเป็นวัตถุที่สามารถหมุนได้ด้วยองศาอิสระ (Degree of Freedom: DOF)
ทั้งหมด 2 องศาอิสระ สามารถจำความรู้นี้มาพิจารณาการใช้คลื่นจากกล้ามเนื้อการกอลกตาในการหาตำแหน่งการกลอกตาได้

\begin{figure}[h]
	\centering
	\includegraphics[width=0.7\textwidth]{images/rem_eog.png}
	\caption{ภาพคลื่น EOG ขณะอยู่ในสถานะการนอนหลับ}
	\hspace{\linewidth}
	\textit{รูปภาพประกอบโดยผู้ใช้ NascarEd \href{https://commons.wikimedia.org/wiki/file:sleep\_stage\_rem.png}{บนโครงการวิกิมีเดีย คอมมอนส์}
		สัญญาอนุญาต Creative Commons Attribution-Share Alike 3.0 Unported}
\end{figure}

การวัดคลื่น EOG สามารถทำได้ด้วยการติดอิเล็กโทรดจำนวน 2 คู่ เพื่อวัดการกลอกตาในแนวระนาบ (yaw) และการกลอกตาในแนวดิ่ง (pitch)
โดยค่าที่อ่านได้จากอิเล็กโทรดหนึ่งคู่จะเป็นการกลอกตาในทิศทางหนึ่ง กล่าวคือการวัด EOG มุ่งสนใจความต่างศักย์ไฟฟ้าของอิเล็กโทรดคู่นั้น
โดยการกลอกตาไปทางซ้าย (หรือกาารกลอกตาขึ้น) จะให้ทิศทางของตวามต่างศักย์ไฟฟ้าที่ต่างจากการกลอกตาไปทางขวา (หรือการกลอกตาลง)

\section{การวัดคลื่นไฟฟ้าบนร่างกาย}

อุปกรณ์วัดคลื่นไฟฟ้าบนร่างกายโดยส่วนมากมักเป็นอุปกรณ์ทางการแพทย์ (medical equipments) ซึ่งต้องมีความแม่นยำและ
ความถูกต้องสูง เนื่องจากความถูกต้องในการอ่านค่าเพื่อวินิจฉัยโรคเป็นสิ่งสำคัญ จะผิดพลาดไม่ได้

อย่างไรก็ดี กระแสของ "นักสร้าง" (makers) ในกลุ่มนักพัฒนาและผู้สนใจในเทคโนโลยี ทำให้การเข้าถึงอุปกรณ์ดังกล่าวเป็นไปได้ง่ายขึ้น
เนื่องด้วยมีความพยายามในการสร้างฮาร์ดแวร์เปิดซอร์ส (open-source hardware) สำหรับวัดคลื่นดังกล่าว

\subsection{โอเพนบีซีไอ (OpenBCI)}

(Todo: OpenBCI image)
 
โอเพ่นบีซีไอ (OpenBCI) เป็นฮาร์ดแวร์แบบเปิดซอร์ส (open-source harware) สำหรับการวัดค่าทางชีวภาพ (biosensing)
อันได้แก่ค่าศักย์ไฟฟ้าในร่างกายมนุษย์ ตัวฮาร์ดแวร์เปิดแบบผังออกมาในลักษณะเดียวกับที่อาร์ดูโน่ (Arduino)
เปิดผังการออกแบบวงจรเป็นสาธารณะ

โอเพ่นบีซีไอสามารถใช้ในการวัดค่าความต่างศักย์ไฟฟ้าซึ่งเกิดจากทั้งสมอง (EEG) กล้ามเนื้อ (EMG) และหัวใจ (EKG)
โดยตัวบอร์ดเป็นวงจรตรรกะแบบ 32 บิทซึ่งใช้ชิป ADS1229 สำหรับการวัดค่าไฟฟ้าร่างกาย ผลิตโดยบริษัท Texas Instruments
และรองรับการวัดช่องสัญญาณได้สูงสุด 8 ช่องสัญญาณพร้อมกัน

\subsection{อิเล็กโทรด (Electrode)}

\begin{figure}[h]
    \centering
    \includegraphics[width=0.5\textwidth]{images/IMG_20190610_155141.jpg}
    \caption{จากบนลงล่าง: อิเล็กโทรดแบบถ้วยทอง (Gold Cup) พร้อมสายสีเหลือง, หูฟัง, และอิเล็กโทรดแบบเจง (Gel) บริเวณติ่งหู}
\end{figure}

อิเล็กโทรดคือขั้วไฟฟ้าที่ทำการติดบนผิวหนังเพื่อวัดค่าศักย์ไฟฟ้าบริเวณจุดต่างๆ ที่สนใจ โดยมากมักใช้อิเล็กโทรดแบบ\textbf{ถ้วยทอง}
(Gold cup electrodes) และแบบ\textbf{เจล} (Get electrodes) ซึ่งสามารถเทียบกันได้ดังนี้

\begin{table}[h]
    \begin{tabularx}{\textwidth}{l|X|X}
         & แบบถ้วยทอง & แบบเจล\\
        \hline
        การทำความสะอาดผิวหนัง  & \multicolumn{2}{l}{ต้องทำความสะอาดผิวหนังก่อน}\\
        \hline
        การติดอิเล็กโทรด & ต้องใช้ยานำไฟฟ้าผิวหนัง (skin conducting paste) เสมอ  & อาจไม่ใช้ยานำไฟฟ้าผิวหนัง (skin conducting paste) แต่การใช้จะให้ผลลัพธ์ที่ดีกว่า \\
        \hline
        ความง่ายในการติด & ติดง่ายกว่าบริเวณที่มีผมหนา & ติดผิวหนังยากกว่าหากมีผมหนา\\
        \hline
        ความเปรอะเปื้อน & มาก เพราะยานำไฟฟ้าผิวหนังจะติดบริเวณผมและศีรษะ & น้อย เจลสามารถลอกออกได้จากผิวหนังโดยไม่ทิ้งคราบ\\
        \hline
        ราคา & ถูกกว่า ใช้ซ้ำได้ & แพงกว่า ส่วนเจลใช้ได้ครั้งเดียว
    \end{tabularx}
\end{table}

\section{การเรียนรู้เชิงลึก (Deep Learning)}
การเรียนรู้เชิงลึก (Deep Learning) คือความพยายามในการจำลองเซลล์ประสามของมนุษย์ให้อยู่ในรูปโมเดลคณิตศาสตร์
ด้วยความเชื่อทางหลักประสาทวิทยา (neurosciences) ว่าความฉลาดของสมองมนุษย์เกิดขึ้นได้จากโครงข่ายประสาทจำนวนมาก
ที่เชื่อมเข้าถึงกัน

\subsection{เปอร์เซปตรอน (Perceptron)}
\begin{figure}[h]
    \includegraphics[width=\textwidth]{images/perceptron.pdf}
    \caption{เปอร์เซปตรอน}
\end{figure}
เปอร์เซปตรอน (Perceptron) เป็นแบบจำลองทางคณิตศาสตร์ของเซลล์สมองหนึ่งเซลล์ โดยมีคุณสมบัติดังนี้

\begin{itemize}
    \item รับเข้าข้อมูลมาในเซลล์จากหลายแหล่ง และให้น้ำหนักกับข้อมูลนั้นต่างกันไป
    \item ส่งออกข้อมูลเพียงค่าเดียว
\end{itemize}

ดังนั้น แบบจำลองทางคณิตศาสตร์สามารถเขียนออกมาจากหลักการสองข้อดังกล่าวได้ด้วยสมการ

$$ y = f\left(W^TX+b\right) $$

เมื่อ $W$ และ $X$ เป็นเวกเตอร์ขนาด $1 \times n$ (โดย $n$ เป็นจำนวนข้อมูลรับเข้า), $b$ เป็นค่าสัมประสิทธิ์คงที่ (ไบแอส: bias)
และ $f$ เป็นฟังก์ชั่นกระตุ้น (activation function)

ยกตัวอย่างการใช้เปอร์เซปตรอนในการแก้ปัญหาอย่างง่ายได้ในที่นี้

\subsubsection{การคาดเดาราคาอสังหาริมทรัพย์}
หากสำรวจราคาอสังหาริมทรัพย์แล้วพบว่า
\begin{itemize}
    \item ราคาอสังหาริมทรัพย์จะเพิ่มขึ้นตามที่ดิน โดยเพิ่มขึ้นทุก 10,000 บาทต่อตารางวา
    \item ราคาอสังหาริมทรัพย์จะเพิ่มขึ้นตามจำนวนห้องนอน โดยเพิ่มขึ้นทุก 200,000 บาทต่อห้องนอน
    \item ราคาอสังหาริมทรัพย์จะลดลงตามจำนวนอายุปี โดยลดลงทุก 7,000 บาทต่ออายุของอสังหาริมทัพย์
\end{itemize}
\noindent
จะสามารถเขียนเปอร์เซปตรอนเพื่อคาดเดาราคาอสังหาริมทรัพย์ได้โดย
$$ y = f\left(W^TX+b\right) $$
เมื่อ $W$ ซึ่งเป็นค่าสัมประสิทธิ์แสดงถึงความสัมพันธ์ข้อมูลรับเข้า ซึ่งเขียนได้จากความสัมพันธ์ดังแสดงด้านล่าง
$$
    W^T = \begin{bmatrix}
        10000 & 200000 & -7000
    \end{bmatrix}
$$
$b$ เป็นไบแอส, จะสมมติให้ $b = 500000$ (ราคาตั้งต้นของบ้าน 0 ห้องนอน พื้นที่ 0 ตารางวา อายุ 0 ปี) 
และ $f(x) = x$ กล่าวคือเป็นฟังก์ชั่นเชิงเส้น

หากต้องการคาดเดาราคาบ้านที่มี 3 ห้องนอน เนื้อที่ 100 ตารางวา และมีอายุ 7 ปี จะสามารถเขียนเวกเตอร์ $P$ ได้เป็น

$$
    X = \begin{bmatrix}
        3 \\
        100 \\
        7
    \end{bmatrix}
$$
และผลการทำนายราคาบ้านคำนวนได้จาก
$$
    \begin{aligned}
        y &= f\left(W^TX+b\right)\\
        &= f\left(\begin{bmatrix}
            10000 & 200000 & -7000
        \end{bmatrix} \times \begin{bmatrix}
            3 \\
            100 \\
            7
        \end{bmatrix}\right)\\
        &= f(30000 + 20000000 + (-49000)) = f(19981000)\\
        &= 19981000
    \end{aligned}
$$

\subsubsection{การสร้างประตูสัญญาณตรรกะด้วยเปอร์เซปตรอน}
เราสามารถสร้างประตูสัญญาณตรรกะ (logic gates) บางชนิดได้ด้วยเปอร์เซปตรอน เช่นการสร้าง AND และ OR gate

ยกตัวอย่างโครงสร้างของ AND gate ซึ่งสามารถสร้างได้ด้วยการกำหนดให้
\begin{itemize}
    \item $X$ เป็นเมทริกซ์ขนาด $1 \times 2$ กล่าวคือเมื่อรับค่า $x_1, x_2$ เป็นค่า 0 หรือ 1 แทนสัญญาณจริงหรือเท็จแล้ว
        $$X = \begin{bmatrix}
            a_1 \\
            a_2
        \end{bmatrix}$$
    \item กำหนดค่าของเมทริกซ์ $W$ เป็น
        $$W^T = \begin{bmatrix}
            1 && 1
        \end{bmatrix}$$
    \item กำหนดค่าของไบแอส $b = -2$
    \item กำหนดฟังก์ชั่น $f(x)$ เป็น step function กล่าวคือ
    $$ f(x) = 
    \begin{cases} 
        0 & \textrm{เมื่อ } x < 0 \\
        1 & \textrm{เมื่อ } x\geq 0
    \end{cases}
    $$
\end{itemize}
และการสร้าง OR gate สามารถทำได้ในลักษณะเดียวกันโดยเปลี่ยน $b$ เป็น $b = -1$

\subsection{เปอร์เซปตรอนแบบหลายชั้น (Multi Layer Perceptron)}

เราอาจสังเกตว่าเปอร์เซพตรอนหนึ่งตัวนั้นทำหน้าที่ได้เพียนแยก (classify) หรือถดถอย (regress) ปัญหาที่เป็นปัญหาเชิงเส้น (linear problems) ได้เท่านั้น อย่างไนก็ตามหากเรากำหนดให้ฟังก์ชั่น $f$ เป็นฟังก์ชั่นที่ไม่ใช่ฟังก์ชั่นเส้นตรงแล้ว เราอาจสร้าง\textbf{เปอร์เซปตรอนแบบหลายชั้น} (Multi Layer Perceptron) ขึ้นมาได้โดยมีลักษณะดังนี้

\chapter{ระเบียบวิธีดำเนินงาน}
\section{การออกแบบการทดลอง}

\subsection{การทดลองเบื้องต้น}

\chapter{ผลลัพธ์และการวิเคราะห์ผล}
\input{redacted/results.tex}

\chapter{บทสรุป}

\section{สรุปผลการปฏิบัติงาน}

การปฏิบัติงานในครั้งนี้เป็นไปได้ตามความคาดหวังของนิสิตและสถานประกอบการ แม้จะมีข้อจำกัดทางด้านระยะเวลาที่ทำให้ไม่สามารถออกแบบและทำการทดลองแบบเต็มรูปแบบได้ แต่การทดลองในเบื้องต้น (initial findings) นั้นเพียงพอที่จะทำไปต่อยอดเป็นหัวข้อโครงงานวิศวกรรมคอมพิวเตอร์ต่อไป

\section{สิ่งที่คาดหวัง}
\begin{itemize}
    \item \textbf{เพื่อเพิ่มพูนประสบการณ์ในการเรียนรู้ที่ไม่อาจหาได้ในห้องเรียน}: เป็นไปตามที่คาดหวัง, ได้ทำงานในสภาวะงานจริง ได้เขียนโค้ดในสภาวะที่มีแรงกดดัน และไม่ใช่การเขียนเป็นงานอดิเรก (hobby) ที่สามารถหยุดได้กลางทาง
    \item \textbf{เพื่อพัฒนาทักษะการทำงาน การสื่อสาร และทักษะ soft skills อื่นๆ}: เป็นไปมากกว่าที่คาดหวัง, ได้รับมอบหมายให้ดูแลทีมวิจัยความง่วง ซึ่งต้องใช้ทักษะในการแก้ปัญหา ตัดสินใจ และสื่อสารกับผู้อื่นอย่างมาก
    \item \textbf{เพื่อเป็นการเตรียมตัวในการทำโครงงานวิศวกรรมคอมพิวเตอร์ และเป็นการเตรียมตัวเขียนวารสารทางวิชาการ}: เป็นไปตามที่คาดหวัง, ขณะนี้ส่งบทความในวารสารวิชาการ รอการตีพิมพ์ 1 ชิ้นบนวารสาร IEEE Access และกำลังจะส่งการประชุมอีก 1 ชิ้น
\end{itemize}

\section{ประโยชน์ที่ได้รับจากการปฏิบัติงาน}
\begin{itemize}
    \item \textbf{ประโยชน์ต่อตนเอง}: เป็นการเพิ่มพูนทักษะและเตรียมพร้อมต่อสายงานวิชาการ และการทำโครงงานวิศวกรรมคอมพิวเตอร์
    \item \textbf{ประโยชน์ต่อสถานประกอบการ}: สามารถช่วยผลิตผลงานทางวิชาการ ตามเป้าหมายและความมุ่งมั่นของสถาบันวิทยสิริเมธี
    \item \textbf{ประโยชน์ต่อมหาวิทยาลัย}: เป็นการสร้างความเชื่อมั่นต่อคณะวิศวกรรมศาสตร์ มหาวิทยาลัยเกษตรศาสตร์ เพิ่มเติมจากที่ศิษย์เก่าของมหาวิทยาลัยเกษตรศาสตร์ได้รับการยอมรับจากสถาบันวิทยสิริเมธี
\end{itemize}

\section{การวิเคราะห์สวอท (SWOT)}
\subsection{ปัจจัยภายในที่เอื้อประโยชน์ (Strength)}

องค์ความรู้เก่าที่มี และความสามารถในการเรียนรู้เพิ่มเติม ทำให้สามารถตัดสินใจและทำงานในทั้งส่วนที่มีประสบการณ์และไม่มีประสบการณ์มาก่อน

\subsection{ปัจจัยภายในที่ส่งผลกระทบ (Weakness)}

ความสามารถในการรับแรงกดดัน ความสามารถในการปรับตัวเข้ากับสถานการณ์

\subsection{ปัจจัยภายนอกที่เอื้อประโยชน์ (Opportunities)}

สถาบันฯ และบุคลากร พร้อมให้โอกาส คำปรึกษา และการสนับสนุนทางวิชาการที่แข็งแกร่ง การเบิกซื้ออุปกรณ์ตามความต้องการเป็นไปได้อย่างไม่ยากลำบาก และค่าตอบแทนนิสิตฝึกงานนั้นเป็นธรรมกับนิสิตฝึกงาน

\subsection{ปัจจัยภายนอกที่ส่งผลกระทบ (Threats)}

เวลาฝึกงานที่จำกัดทำให้ไม่สามารถออกแบบการทดลองระยะยาวตามที่คาดหวังได้
\section{ความประทับใจพิเศษ}

เนื่องจากเป็นการฝึกงานเป็นปีที่สอง ทำให้คุ้นเคยกับบุคลากรที่สถาบัน การปรับตัวเข้ากับที่ฝึกงานจึงกินเวลาไม่นานมาก เมื่อพิจารณาประกอบกับสวัสดิการของสถาบันฯ และความห่วงใย รวมถึงความเอาใจใส่ของบุคลากร ทำให้มั่นใจว่าสถาบันฯ พร้อมจะสนับสนุนสุขภาวะการทำงานที่ดีควบคู่กับความมุ่งมั่นในความเป็นเลิศทางวิชาการ

\chapter{ปัญหาและข้อเสนอแนะ}

\section{นิสิต}
\subsection{ปัญหา}

\begin{itemize}
    \item นิสิตมีความไม่มั่นใจในทักษะ ศักยภาพ และความรู้ความสามารถตนเอง ว่าจะเพียงพอกับการฝึกงานหรือไม่ แม้ว่าในความจริงจะมีมากพอก็ตาม
\end{itemize}

\subsection{ข้อเสนอแนะ}

\begin{itemize}
    \item การจัดการกับ imposter syndrome หรือการไม่มั่นใจในตัวนิสิตเอง จำเป็นจะต้องได้รับการสนับสนุนจากทุกภาคส่วน ทั้งนิสิต ภาควิชา คณะ และหน่วยให้คำปรึกษามหาวิทยาลัย
\end{itemize}

\section{สถานประกอบการ}
\subsection{ปัญหา}

\textit{(ไม่มี)}

\subsection{ข้อเสนอแนะ}

\textit{(ไม่มี)}

\section{มหาวิทยาลัย}

\subsection{ปัญหา}

\begin{itemize}
    \item มหาวิทยาลัยเลื่อนกำหนดการเปิดภาคเรียน ทำให้นิสิตไม่สามารถมาเรียนได้ในสองถึงสามสัปดาห์แรก
    \item มหาวิทยาลัยไม่ใส่ใจนิสิตฝึกงานต่างประเทศ และในสถานประกอบการต่างจังหวัด แม้จะให้ขาดเรียนได้โดยไม่นับเป็นการขาดเรียน แต่การดำเนินการเอกสารหลายๆ อย่างต้องมาทำที่มหาวิทยาลัย
\end{itemize}

\subsection{ข้อเสนอแนะ}

\begin{itemize}
    \item จัดหาอธิการบดี สภามหาวิทยาลัย และผู้บริหารที่มีความเข้าใจในนิสิตและบริหารงานอย่างมืออาชีพได้ดีกว่านี้
\end{itemize}


\bibliographystyle{ieeetr}
\bibliography{citations}

\begin{appendices}
\renewcommand{\thechapter}{\thaiAlph{chapter}}
    
\chapter{บันทึกประจำวัน}
\section*{4/6/2562}

เนื่องจากเข้าทำงานเป็นวันแรก จึงต้องจัดสถานที่ทำงาน และทำงานต่อจากที่ได้รับมอบหมายก่อนการฝึกงาน

งานที่ได้รับมอบหมายโดยคร่าวคือการวิเคราะห์สภาวะความง่วงในคน โดยศึกษาจากกลุ่มเป้าหมายของพนักงานบริษัท
ซึ่งอาจารย์ที่ปรึกษาให้สิทธิ์ในการกำหนดแนวทางการวิเคราะห์ได้โดยอิสระ อย่างไรก็ตามงานของการวิเคราะห์ความง่วง
โดยตั้งต้นนั้นมักใช้การวิเคราะห์ภาพจากดวงตา (gaze monitoring) ซึ่งใช้วันนี้ในการหางานวิจัยตั้งต้น

นอกจากนี้ยังศึกษาแนวทาง ข้อกำหนด และมาตรฐานจริยธรรมในการทดลองภายในมนุษย์ (human subject research)

\section*{5/6/2562}

ศึกษาแนวทางในการทำ eye gazing ตามหนังสือที่ได้รับมอบหมาย และนำเสนองานวิจัยต่อจากที่เลือกจากเมื่อวาน

ปรับแก้แนวทางในการวิจัย และได้รับมอบหมายให้ออกแบบวิธีการทดลองโดยคร่าว
หารือกับทีมโปรแกรมเมอร์ว่าด้วยซอฟต์แวร์สำหรับการทดลอง

\section*{6/6/2562}

ศึกษาอุปกรณ์สำหรับติดตามดวงตา (Gazepoint) ก่อนจะพบว่าอุปกรณ์มีข้อจำกัดในการทำงานบางส่วน ทำให้ไม่สามารถดึง
ภาพดวงตาออกมาใช้ในโปรแกรมภายนอกได้ และติดต่อกับผู้ผลิตอุปกรณ์เพื่อหารือความเป็นไปได้ในการดึงภาพดวงตา

ศึกษาการใช้ Pytorch ในการทำการเรียนรู้เชิงลึก (deep learning) แทนที่ Keras

\section*{7/6/2562}

เปลี่ยนแนวทางการทำวิจัยด้วยข้อจำกัดของอุปกรณ์ มาเป็นการทำวิจัยบนกล้ามเนื้อตา (EOG) ค้นคว้าและทบทวนวรรรณกรรม
ที่เกี่ยวข้องกับงาน

ทำแบบทดสอบสำหรับบทเรียนจริยธรรมการวิจัยในมนุษย์จนอยู่ในเกณฑ์ได้รับประกาศนียบัตรผ่านการอบรม

\chapter{ภาพถ่ายสถานที่ปฏิบัติงาน}

\section*{ภาพการฝึกงานวันที่ 4/6/2562}
\begin{figure}[H]
    \centering
    \includegraphics[width=0.5\textwidth]{/home/srakrn/Works/senior/internship/internship_report/diary/images/IMG_20190604_152401.jpg}
    \caption{สถานที่ทำงานหลังจากจัดที่ทำงานแล้ว}
\end{figure}

\section*{ภาพการฝึกงานวันที่ 5/6/2562}
\begin{figure}[H]
    \centering
    \includegraphics[width=0.5\textwidth]{/home/srakrn/Works/senior/internship/internship_report/diary/images/IMG_20190605_112217.jpg}
    \caption{หนังสือที่ได้รับมอบหมายให้อ่านและศึกษา ถ่ายคู่กับสไลด์สรุปงานวิจัย}
\end{figure}

\section*{ภาพการฝึกงานวันที่ 6/6/2562}
\begin{figure}[H]
    \centering
    \includegraphics[width=0.5\textwidth]{/home/srakrn/Works/senior/internship/internship_report/diary/images/IMG_20190606_161925.jpg}
    \caption{อุปกรณ์สำหรับติดตามดวงตา Gazepoint ขณะกำลังจับม่านตา}
\end{figure}

\section*{ภาพการฝึกงานวันที่ 10/6/2562}
\begin{figure}[H]
    \centering
    \includegraphics[width=0.5\textwidth]{/home/srakrn/Works/senior/internship/internship_report/diary/images/IMG_20190610_155035.jpg}
    \caption{สมาชิกทีม Drowsiness Research และสมาชิกทีม BRAIN ขณะทดสอบสมมติฐาน}
\end{figure}

\section*{ภาพการฝึกงานวันที่ 24/6/2562}
\begin{figure}[H]
    \centering
    \includegraphics[width=0.5\textwidth]{/home/srakrn/Works/senior/internship/internship_report/diary/images/IMG_20190624_153749.jpg}
    \caption{อาจารย์ธีรวิทย์ วิไลประสิทธิ์พร ขณะทบทวนงานวิจัยโดยคร่าว}
\end{figure}

\end{appendices}

\end{document}