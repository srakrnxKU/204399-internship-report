% !TEX program = xelatex

\documentclass[16pt,a4]{internshipreport}

\usepackage{lipsum}  
\usepackage[numbib]{tocbibind}

\title{การเขียนรายงานฝึกงานที่เขียนเทมเพลทนานกว่ารายงาน}
\entitle{Writing a nonsensible internship report just to test the template}
\location{ครัวป้าวง ป่ายุบใน}
\date{2562}
\author{ศิระกร ลำใย}
\studentid{5910500023}

\begin{document}

\maketitle

\section*{บทคัดย่อ}
\addcontentsline{toc}{section}{บทคัดย่อ}
\lipsum[2-4]

\section*{กิตติกรรมประกาศ}
\addcontentsline{toc}{section}{กิตติกรรมประกาศ}
ประการแรก ข้าพเจ้าขอขอบพระคุณครอบครัวที่สนับสนุนและตัดสินใจให้ทำในสิ่งที่เชื่อว่าถูกต้องมาโดยตลอด
การให้อิสระทางความคิดและการตัดสินใจนั้นเป็นสิ่งที่จำเป็น มีค่าอย่างยิ่ง และสร้างเสริมประสบการณ์ที่แข็งแรง
ในการตัดสินใจทำหลายๆ อย่าง

ขอขอบพระคุณอาจารย์ธีรวิทย์ วิไลประสิทธิ์พร จากสถาบันวิทยสิริเมธี และอาจารย์ธนาวินท์ รักธรรมานนท์
จากมหาวิทยาลัยเกษตรศาสตร์ สำหรับการสนับสนุนในการฝึกงานทั้งในช่วงปี 2561 และปี 2562 (กล่าวคือในสองปีที่ผ่านมา)

ขอขอบพระคุณอาจารย์จากภาควิชาวิศวกรรมคอมพิวเตอร์ และวิศวกรรมไฟฟ้า มหาวิทยาลัยเกษตรศาสตร์
ผู้อยู่เบื้องหลังทัศนะคติบวก ผู้ขัดเกลามุมมองต่อโลกใบนี้ และมอบองค์ความรู้จำนวนมาก: อาจารย์จิตร์ทัศน์ ฝักเจริญผล,
อาจารย์ชัยพร ใจแก้ว, อาจารย์ธนาวินท์ รักธรรมานนท์, อาจารย์ภารุจ รัตนวนพันธุ์, อาจารย์จเร เลิศสุดวิชัย,
อาจารย์กาญจนพันธุ์ สุขวิชชัย และอื่นๆ ที่มิอาจเอ่ยนามได้หมด

ขอบคุณเป็นอย่างยิ่งในไมตรีจิตจากนิสิตและเจ้าหน้าที่ที่สถาบันวิทยสิริเมธี ผู้ซึ่งเป็นทั้งรุ่นพี่และผู้ร่วมงานที่น่ารัก: สมบัติ เกตุรัตน์
กฤษณี คำทวี, นรินทร์ คุณาเศรษฐ, ณกรณ์ ขำชัยสีเมฆ, ภุชงค์ สร้อยสุดารัตน์, (พี่ก้อง), (พี่คนอื่น)\dots

ขอขอบคุณ (หรือหากกล่าวด้วยสำนวนของข้าพเจ้า, กราบ) เพื่อนนิสิตมหาวิทยาลัยเกษตรศาสตร์ ทั้งที่ให้กำลังใจในวันที่ท้อถอย
กดดันให้พัฒนาตนเอง และมอบความมุ่งมั่นอันเต็มเปี่ยมให้: รวิสรา รัตนวรรณนุกุล, กรวิชญ์ ชัยกังวาฬ,
กิตติยา กู้เกียรติกูล, ณัฐณิชา ช้างจันทร์, วรชัย วุฒิวรชัยรุ่ง, สิรภพ กางกั้น, พรมนัส หอมเกสร,
ณัฐพงศ์ สมบูรณ์ภัทรกิจ, ปิยวัช ภวะจันทร์สถิตย์, ณัฐพล เวชกิจวาณิชย์, คุณานนต์ บุรเทพ, ณิชา ลิ้มมณี
และทุกท่านที่มิอาจเอ่ยนามได้หมด (อีกครั้ง)

ขอบคุณสมาชิก (ผู้ป้วนเปี้ยนในห้อง) กลุ่มวิจัยเชิงทฤษฎีสำหรับบรรยากาศที่สนุกยิ่งในการพักผ่อน:
ณัฐวุฒิ เพ็ชรมาก, นนทพัทธ์ วงศ์วัฒนากิจ, พงศกร อัจฉริยศักดิ์ชัย, ชยุตพงศ์ พรมภักดิ์, มณฑล จรัสตระกูล,
ภิญญพร เอี่ยมมงคล, ชวิน เอี่ยมวรวุฒิกุล, (พี่แหมง), (พี่บลู) และที่ไม่กล่าวถึงมิได้
คืออาจารย์จิตร์ทัศน์ ฝักเจริญผล อาจารย์หัวหน้ากลุ่มวิจัย

ขอบคุณวงดนตรีทุกแห่งและสมาชิกวงดนตรีทุกท่านที่ช่วยขับเคลื่อนสุนทรียศาสตร์ในการทำงาน:
วงดุริยางค์ฟิลฮาร์โมนิกแห่งประเทศไทย, วงดุริยางค์ซิมโฟนีแห่งลอนดอน, เบอร์ลินฟิลฮาร์โมนิก,
บีเอ็นเคโฟร์ตี้เอท (โดยเฉพาะสมาชิก: เฌอปราง อารีย์กุล, จิรดาภา อินทจักร, แพรวา สุธรรมพงษ์,
เจนนิษฐ์ โอ่ประเสริฐ และมณิภา รู้ปัญญา), ฟีเวอร์ (โดยเฉพาะสมาชิก: กมลพร โกสียรักษ์วงศ์,
จิรัชญา จันทร์จิเรศรัศมี, ปาลีรัตน์ ก้อนบาง, นภัสพร ศรีประภา และรัทยา ผลเกิด), ไปส่งกูบขส.ดู๊,
ชาบลูส์, นายิกา ศรีเนียน, และศิลปินจำนวนมากที่ข้าพเจ้าเสพงาน ผู้ที่มิได้เอ่ยนามมาในที่นี้

การเกิดขึ้นของสถาบันวิทยสิริเมธีจะเป็นไปไม่ได้ หากไม่ได้รับการสนับสนุนจากบริษัทปตท. จำกัด (มหาชน)
พร้อมบริษัทในเครือ และธนาคารไทยพาณิชย์ จำกัด (มหาชน) 
ข้าพเจ้าขอขอบพระคุณในความมุ่งมั่นที่จะเห็นการขับเคลื่อนนโยบายทางวิทยาศาสตร์ของประเทศจากทั้งสองบริษัท
และหวังเป็นอย่างยิ่งว่าการเกิดขี้นของสถาบันฯ จะเป็นแรงสำคัญในการผลักดันประเทศต่อไป

\tableofcontents

\listoffigures

\listoftables

\section{บทนำ}

\subsection{ความสำคัญและที่มา}
หลักสูตรวิศวกรรมศาสตรบัณฑิต สาขาวิชาวิศวกรรมคอมพิวเตอร์ หลักสูตรปรับปรุง พ.ศ 2556
ระบุให้ผู้เรียนทุกคนต้องเข้ารับการฝึกงาน เพื่อเพิ่มพูนประสบการณ์ในการเรียนรู้ที่ไม่อาจหาได้ในห้องเรียน
และเป็นการฝึกทักษะของวิศวกรในการทำงานจริง

คณะวิศวกรรมคอมพิวเตอร์ และภาควิชาวิศวกรรมคอมพิวเตอร์ จึงกำหนดให้มีการเรียนการสอนในรายวิชา
01204399 หรือการฝึกงาน แบ่งเป็นการฝึกงานภาคฤดูร้อนสำหรับนิสิตที่ไม่ได้สหกิจ
และฝึกงานต่อเนื่องในช่วงเวลาของภาคฤดูร้อนและภาคการศึกษาต้นของมหาวิทยาลัยสำหรับนิสิตที่สหกิจ
จึงเป็นที่มาของรายงานเล่มนี้ซึ่งเป็นหนึ่งในข้อกำหนด/ข้อบังคับของการฝึกงาน

\subsection{วัตถุประสงค์การปฏิบัติงาน}
\begin{itemize}
    \item เพื่อเพิ่มพูนประสบการณ์ในการเรียนรู้ที่ไม่อาจหาได้ในห้องเรียน
    \item เพื่อพัฒนาทักษะการทำงาน การสื่อสาร และทักษะ soft skills อื่นๆ
    \item เพื่อเป็นการเตรียมตัวในการทำโครงงานวิศวกรรมคอมพิวเตอร์ และเป็นการเตรียมตัวเขียนวารสารทางวิชาการ
\end{itemize}

\subsection{ขอบเขต}
ไว้มาเขียน

\subsection{ประวัติและรายละเอียดสถานประกอบการ}
\textbf{สถาบันวิทยสิริเมธี (VISTEC)} เป็นบัณฑิตวิทยาลัย (graduate school) ซึ่งมุ่งเน้นความเป็นเลิศในการทำวิจัย 
ตั้งอยู่ในพื้นที่วังจันทร์วัลเลย์ (Wangchan Valley) และเขตนวัตกรรมระเบียงเศรษฐกิจพิเศษภาคตะวันออก
(Eastern Economic Corridor of Innovation: EECi) เลขที่ 555 หมู่ 1 ตำบลป่ายุบใน อำเภอวังจันทร์ จังหวัดระยอง
ก่อตั้งขึ้นเมื่อปี พ.ศ. 2558 โดยมูลนิธิพลังสร้างสรรค์นวัตกรรม ภายใต้การสนับสนุนเงินทุนจากบริษัท
ในกลุ่มของการปิโตรเลียมแห่งประเทศไทย (ปตท.)

VISTEC มุ่งเน้นการจัดการศึกษาด้านวิทยาศาสตร์ วิศวกรรม และเทคโนโลยี โดยมีศูนย์วิจัยวิทยาศาสตร์และเทคโนโลยีชั้นแนวหน้า
(Frontier Research Center) ซึ่งเป็นศูนย์กลาง ในการเสริมสร้างความเข้มแข็งทางการวิจัย
และให้การสนับสนุนด้านทุนการวิจัยแก่สถาบันฯ เป็นศูนย์รวมนักวิจัยที่มีความเชี่ยวชาญสูง ช่วยขับเคลื่อนการดำเนินงานด้านการศึกษา 
ิจัย การสร้างนวัตกรรม สร้างความร่วมมือทางด้านวิจัยกับสถาบันการศึกษา ภาคธุรกิจ ภาคอุตสาหกรรม
และหน่วยงานด้านการวิจัยวิทยาศาสตร์และเทคโนโลยี

\textbf{ห้องปฏิบัติการเบรน} (Bio-inspired Robotics and Neural Engineering: BRAIN)
ณ สำนักวิชาวิทยาศาสตร์และเทคโนโลยีสารสนเทศ สถาบันวิทยสิริเมธี มุ่งเน้นศึกษาการสร้างหุ่นยนต์ที่มีลักษณะร่วมกับกายวิภาค
(anatomy) ของสิ่งมีชีวิต และใช้เทคโนโลยีจำพวก Machine Learning หรือ Deep Learning ในการจำแนก วิเคราะห์
และประมวลผลคลื่นสมองของมนุษย์ เพื่อสร้างส่วนติดต่อผู้ใช้ผ่านสมอง (Brain Controlled Interfaces: BCIs)

ลักษณะงานที่ได้รับผิดชอบจากห้องปฏิบัติการฯ เป็นงานของผู้ช่วยนักวิจัย (Research Assistant: RA) ซึ่งช่วยนิสิตระดับ
บัณฑิตศึกษาในการเตรียมการทดลอง ออกแบบ และพัฒนาเครื่องมือวัดผล ควบคุมการทดลอง และทดสอบสมมติฐานเพื่อตีพิมพ์
องค์ความรู้ในวารสารวิชาการต่อไป

ที่ปรึกษาและผู้ควบคุมการฝึกงานในครั้งนี้ คืออ. ดร. ธีรวิทย์ วิไลประสิทธิ์พร หัวหน้าหน่วยวิจัย (Principal Investigator: PI)
และมีระยะเวลาปฏิบัติงานประมาณ 2 เดือน กล่าวคือตั้งแต่วันที่ 4 มิถุนายน ถึง 31 กรกฎาคม 2562

\subsection{ประโยชน์ที่คาดว่าจะได้รับ}
เป็นการสร้างพื้นฐานในด้านงานวิจัย รวมถึงเตรียมพื้นฐานในการทำโครงงานวิศวกรรมคอมพิวเตอร์ โดยมีความมุ่งหวังจะต่อยอด
งานดังกล่าวเป็นงานวิจัยตีพิมพ์ต่อไป
\end{document}